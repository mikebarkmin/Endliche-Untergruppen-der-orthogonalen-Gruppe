\section{Endliche Gruppen im zweidimensionalen Raum}
\begin{theorem}
 Sei $\mathcal{G}$ eine endliche Untergruppe von $\OR{2}$, dann ist $\mathcal{G}$ entweder eine zyklische Gruppe $\mathcal{C}^n_2$ oder eine Diedergruppe $\mathcal{H}^n_2$ für $n \in \mathbb{N}$
\end{theorem}
\begin{proof}
 Wir nehmen an, dass $\mathcal{H} \leq \mathcal{G}$ durch die Menge aller Drehungen in $\mathcal{G}$ gebildet wird.
 
 Wir wollen zeigen, dass $\mathcal{H}$ zyklisch sein muss. Für $|\mathcal{H}|=1$ ist dieses bereits klar. Wenn $|\mathcal{H}| \neq 1$ wählen wir eine Drehung $R \in \mathcal{H}$, sodass $R \neq E_2$ und der Drehwinkel $\theta(R)$ minimal ist. Wenn wir jetzt eine weitere Drehung $T \in \mathcal{H}$ nehmen, dann können wir ein $m \in \mathbb{Z}$ finden, sodass \begin{align*}
        &m \theta(R)\leq\theta(T)<(m+1)\theta(R) \\
        \Leftrightarrow \ &0 \leq \theta(T)-m\theta(R)<\theta(R) \\
        \Leftrightarrow \ &0 \leq \theta(TR^{-m})<\theta(R)                                                                                                                                                                                                                                                                                                                                                           
\end{align*}
Da wir $\theta(R)$ minimal gewählt haben, gilt $\theta(TR^{-m})=0$. Also muss $TR^{-m}=E_2$ sein und es folgt $T=R^{m}$. Demnach ist $\mathcal{H}$ zyklisch mit $\mathcal{H}=<R>$ ($R$ ist Erzeuger von $\mathcal{H}$). Damit folgt auch, dass $\theta(R)=\frac{2}{n}\pi$, wenn $n=|\mathcal{H}|$. Wenn $\mathcal{G} = \mathcal{H}$ gilt, dann haben wir gezeigt, dass $\mathcal{G}$ zyklisch ist und wir bezeichnen $\mathcal{G}$ mit $\mathcal{C}^n_2$.

Als nächstes nehmen wir dann an, dass $\mathcal{G} \neq \mathcal{H}$ und wählen eine Spiegelung $S \in \mathcal{G}$. Nun wählen wir eine weitere beliebige Spiegelung $T \in \mathcal{G}$ mit $T \neq S$, dann gilt $\det(ST)=\det(S)\det(T)=1$. Deshalb gilt $ST \in \mathcal{H}$ und es handelt sich bei $ST$ um eine Drehung. Dann gilt $T \in S\mathcal{H}$, da $S^{-1}=S$. So gilt, dass $\mathcal{H}$ eine Untergruppe von $\mathcal{G}$ mit Index $2$ ist und $\mathcal{H}=<R>$. Dann gilt $\mathcal{G}=<R,S>$ $=\{E_2,R,\dots \ ,R^{n-1},S,SR,\dots ,SR^{n-1}$\} und $|\mathcal{G}|=2n$. Da $RS$ eine Spiegelung ist, gilt $(RS)^2=E_2$ und $RS=SR^{-1}=SR^{n-1}$, damit sind alle Verknüpfungen in $\mathcal{G}$ festgelegt. Die Gruppe $\mathcal{G}$ bezeichnen wir mit $\mathcal{H}^n_2=<S,R>$ (Die Diedergruppe von Ordnung 2n).                                                                                                                                                                                                                                                                                                                                                           
\end{proof}

