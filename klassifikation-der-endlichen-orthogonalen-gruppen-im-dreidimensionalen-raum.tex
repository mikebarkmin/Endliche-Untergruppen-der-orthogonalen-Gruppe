\subsection{Klassifikation der endlichen orthogonalen Gruppen}
Nachdem wir die endlichen Rotationsgruppen im dreidimensionalen Raum klassifiziert haben, möchten wir nun die endlichen Gruppen klassifizieren. Wir schauen uns zunächst die Gruppe $\mathcal{W}^*$ an. Diese soll alle orthogonalen Abbildungen umfassen, welche den Würfel wieder auf sich selber abbilden. Natürlich ist leicht zu sehen, dass $\mathcal{W}$ kleiner ist als $\mathcal{W}^*$. Aber wir können bemerken, dass für $T \in \mathcal{W}^*\backslash\mathcal{W}$ \ $-T = -1 \cdot T \in \mathcal{W}$ gilt. Daher gilt $\mathcal{W}^*=\mathcal{W}\cup(-1)\mathcal{W}$.
Dieser Eigenschaft lässt sich auf alle Untergruppe von $\mathcal{O}(V)$ generalisieren, egal welche Dimension $V$ besitzt.

\begin{lemma}
    Wenn $\mathcal{G} \leq \mathcal{O}(V)$ und $\mathcal{H}$ Rotationsuntergruppen von $\mathcal{G}$ sind, dann gilt entweder $\mathcal{H}=\mathcal{G}$ oder $[\mathcal{G}:\mathcal{H}] = 2$. Also ist $\mathcal{H}$ ein Normalteiler von $\mathcal{G}$.
\end{lemma}
\begin{proof}
    Sei $T \in \mathcal{G}\backslash\mathcal{H}$. Dann können wir uns ein beliebiges Element $S\in\mathcal{G}\backslash\mathcal{H}$ wählen und es gilt immer $\det(T^{-1}S) = (-1)^2=1$. Daher ist $T^{-1}$ ein Element von $\mathcal{H}$ und es gilt $\mathcal{G} = \mathcal{H} \cup T\mathcal{H}$ und $[\mathcal{G}:\mathcal{H}]=2$.
\end{proof}
Nachdem wir das Lemma bewiesen haben, können wir uns nun an den finalen Satz dieser Ausarbeitung machen. Wir möchten nun die endlichen Untergruppen der orthogonalen Abbildungen im dreidimensionalen Raum klassifizieren.

Angenommen $\mathcal{H} \leq \mathcal{G}$ sei eine Untergruppe bestehend aus Rotationen von $\mathcal{G}$. Um eine Einteilung zu schaffen brachten wir drei Fälle.
 
 
 \textbf{Fall 1:} Falls $\mathcal{G} = \mathcal{H}$, dann können wir uns auf die Klassfizierung der endlichen Rotationsgruppen berufen und sind fertig. Wenn dieses nicht der Fall ist, dann hat $\mathcal{H}$ den Index $2$ in $\mathcal{G}$.
 
 
 \textbf{Fall 2:} Sei $-E_3 \in \mathcal{G}$, dann gilt $\mathcal{G} = \mathcal{H} \cup - \mathcal{H}$ und $\mathcal{G}$ besteht aus einer Untergruppe bestehend aus Rotationen und ihren Negativen.


 \textbf{Fall 3:} Sei $\mathcal{G}\neq \mathcal{H}$, $-E_3 \notin \mathcal{G}$ und $R\mathcal{H}$ ist eine von $\mathcal{H}$ verschiedene Nebenklasse in $\mathcal{G}$. Dann ist $R^2\in\mathcal{H}$, wenn $R^2 \in R \mathcal{H}$ ist und dieses würde bedeuten, dass $R \in \mathcal{H}$. Da $\mathcal{H}$ ein Normalteiler von $\mathcal{G}$ ist folgt, dass $(-R\mathcal{H})(-R\mathcal{H})=R^2\mathcal{H}=\mathcal{H}$. Daher gilt, dass die Menge $\mathcal{K} = \mathcal{H} \cup (-R)\mathcal{H}$ eine Gruppe bestehend aus Rotationen von $\OR{3}$ ist. Dann gilt, dass $\mathcal{G} = \mathcal{H} \cup \{-T:T\in\mathcal{K}\backslash\mathcal{H}\}$ eine Untergruppe von $\OR{3}$ ist, deren Rotationsuntergruppe $\mathcal{H}$ ist. Damit können wir auch die dritte Art von Gruppen klassifizieren, indem wir eine Gruppe der ersten Klasse nehmen, die eine Untergruppe $\mathcal{H}$ von Rotationen mit Index 2 besitzt und dann $\mathcal{G}=\mathcal{H}\cup\{-T:T\in\mathcal{K}\backslash\mathcal{H}\}$ setzten. Die Gruppe bezeichnen wir dann mit $\mathcal{K}]\mathcal{H}$.


\begin{theorem}
 Sei $\mathcal{G}\leq\OR{3}$, dann ist $\mathcal{G}$ isomorph zu einer der folgenden Klassen:
 \begin{itemize}
  \item $\mathcal{C}^n_3,n\geq1;\mathcal{H}^n_3,n\geq2;\mathcal{T};\mathcal{W};\mathcal{I}$
  \item $(\mathcal{C}^n_3)^*,n\geq1;(\mathcal{H}^n_3)^*,n\geq 2;\mathcal{T}^*;\mathcal{W}^*;\mathcal{I}^*$
  \item $\mathcal{C}^{2n}_3]\mathcal{C}^n_3,n\geq1;\mathcal{H}^n_3]\mathcal{C}^n_3,n\geq 2;\mathcal{H}^{2n}_3]\mathcal{H}^n_3,n\geq2;\mathcal{W}]\mathcal{T}$
\end{itemize}
Hinweis: $\mathcal{R}^*:=\mathcal{R}\cup -\mathcal{R}$ und $\mathcal{R}]\mathcal{P}:=\mathcal{P}\cup \{-U|U\in \mathcal{R} \backslash \mathcal{P} \}$
\end{theorem}
