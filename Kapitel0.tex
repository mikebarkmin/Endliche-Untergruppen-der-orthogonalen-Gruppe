\section{Einleitung}
In diesem Vortrag werden die endlichen Untergruppen im zwei- und dreidimensionalen Raum klassifiziert. Um die nachfolgenden Sätze verstehen zu können ist es notwendig grundlegende Gruppentheoriebegriffe zu kennen. Ein besonderes Augenmerk soll vor allem auf die nachfolgenden Definition gelegt werden, da diese für den Vortrag essentiell sind.
\begin{defi}[Zyklische Gruppe]
 Eine Gruppe $\mathcal{C}$ heißt zyklisch, wenn sie ein Element $A$ enthält, sodass für jedes Element $B$ von $\mathcal{C}$ gilt $A^n = B$ mit $n \in \mathbb{Z}$. Außerdem gilt, dass $\mathcal{C}$ die einzige Untergruppe von $\mathcal{C}$ ist die $A$ enhält. Dann nennen wir $A$ den Erzeuger von $\mathcal{C}$ und können schreiben $\mathcal{C} = <A>$.
\end{defi}
\begin{defi}[Diedergruppe]
 Eine Gruppe $\mathcal{H}$ ist eine Diedergruppe, wenn sie die Isometriegruppe eines regelmäßigen $n$-Ecks in der Ebene ist. Sie besteht dann aus $n$ Drehungen und $n$ Spiegelungen. Also aus insgesamt $2n$ Elementen.
\end{defi}
\begin{defi}[Invarianter Unterraum]
 Ein Unterraum $W \leq V$ einer linearen Abbildung $T:V \rightarrow V$ heißt invarianter Unterraum, wenn $T(W) \subseteq W$ gilt. Man sagt, dass $W$ invariant unter $T$ ist. Wenn dies gilt, dann können wir $T$ auf $W$ einschränken, um eine neue lineare Abbildung zu erhalten. $T|W:W\rightarrow W$.
\end{defi}
