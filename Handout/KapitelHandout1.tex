\section{Orthogonale Transformationen im zweidimensionalen Raum}
\begin{bem}
 Sei $T \in \mathcal{O}(\mathbb{R}^2)$, dann ist $T$ eindeutig über die Basisvektoren $e_1=(1,0)$ und $e_2=(0,1)$ definiert. Da $T \in \mathcal{O}(\mathbb{R}^2)$ und somit längen- und orthogonalitätserhaltend ist, existiert ein eindeutiges $\theta \in [0,2 \pi)$, sodass $Te_1=(\cos(\theta),\sin(\theta))$ und $Te_2=\pm(-\sin(\theta),\cos(\theta))$.
 
 Wenn $Te_2=(-\sin(\theta),\cos(\theta))$, dann wird $T$ durch die Matrix 
 \begin{center}
  $A = \begin{pmatrix}
        \cos(\theta) && -\sin(\theta) \\
        \sin(\theta) && \cos(\theta)
       \end{pmatrix}$

 \end{center}
repräsentiert und $T$ ist eine Drehung um den Ursprung mit dem Winkel $\theta$. Außerdem gilt $\det T = \det A = \cos^2(\theta) + \sin^2(\theta) = 1$

Wenn $Te_2=(\sin(\theta),-\cos(\theta))$, dann wird $T$ durch die Matrix 
 \begin{center}
  $A = \begin{pmatrix}
        \cos(\theta) && \sin(\theta) \\
        \sin(\theta) && -\cos(\theta)
       \end{pmatrix}$

 \end{center}
repräsentiert und $T$ ist eine Spiegelung. Außerdem gilt $\det T = \det A = -1$. 
\end{bem}
\begin{theorem}
 Jede orthogonale Transformation im zweidimensionalen Raum ist entweder eine Spiegelung oder eine Drehung.
\end{theorem}
