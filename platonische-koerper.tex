\subsection{Platonische Körper}
In den vorherigen Kapiteln haben wir uns mit der Klassifizierung der endlichen Untergruppen der orthogonalen Gruppen im Zweidimensionalen beschäftigt. Dabei haben wir feststellen können, dass es für die Klassifizierung ausreicht nur Gruppen zu betrachten, die zu reguläre Polygonen gehören. Wie zum Beispiel die Gruppe $\mathcal{C}_4$ die zu einem Quadrat gehört. Da wir uns nun damit beschäftigen wollen, wie wir die endlichen Untergruppen der orthogonalen Gruppe im Dreidimensionalen klassifizieren können, liegt es nahe sich mit regulären Polyedern zu befassen.\\
Wir beginnen mit einigen Definitionen: 
\begin{defi}[Einfaches Polyeder]
	Ein Polyeder heißt einfach, wenn seine Oberfläche sich stetig in eine Kugelfläche überführen lässt, das heißt einfach Polyeder haben keine \enquote{Löcher} \citep[211]{Mainzer1988}
\end{defi}
\textbf{Beispiele für ein einfaches und ein nicht-einfaches Polyeder}
\begin{defi}[Konvexes Polyeder]
	Ein Polyeder heißt konvex, wenn zu je zwei Punkten aus dem Inneren des Polyeders auch deren Verbindungsstrecke ganz im Polyeder liegt. \citep[51]{Mueller2012}
\end{defi}
\begin{defi}[Reguläres Polyeder]
	Ein konvexes Polyeder heißt regulär, wenn alle Flächen zueinander kongruente regelmäßige Vielecke sind und in jeder Ecke gleich viele Vielecke zusammenstoßen. \citep[51]{Mueller2012}
\end{defi}
Diese Definitionen werden uns erneut im Beweis des Eulerschen Polyedersatzes begegnen. Es gibt genau fünf reguläre Polyeder.\\
\textbf{Bild}\\
Für diesen Sachverhalt wurden im Verlaufe der Geschichte verschiedene Beweise formuliert, von denen wir zwei nachvollziehen und schließlich gegenüberstellen wollen. \\
Wir werfen nun zunächst einen Blick zurück in die Geschichte und befassen uns mit der Entdeckung der fünf reguläre Polyeder und ihrer Bedeutung im Laufe der Jahrhunderte. \\
Bereits im antiken Griechenland wurde erkannt und bewiesen, dass nur fünf platonische Körper konstruiert werden können: Tetraeder, Hexaeder (Würfel), Oktaeder, Ikosaeder und Dodekaeder. Ihr Namensgeber Platon beschrieb die regulären Polyeder etwa 350 v. Chr. im Kontext mit naturphilosophischen, kosmologischen und in seinem Werk "Timaios". Platon war nicht nur von der Vollkommenheit und Symmetrie beeindruckt, sondern insbesondere davon überzeugt, dass vier der platonischen Körper die Strukturen der Elemente Feuer, Wasser, Erde und Luft beschreiben, aus denen der Kosmos besteht. Der Dodekaeder repräsentiert hingegen die göttlich gegebene Konstellation der Himmelskörper.\\
Die erste Niederschrift der Konstruktion mit Lineal und Zirkel findet sich in Euklids mehrbändigen Werk Elemente (Buch XII), das um 300 v.Chr. entstand. Von ihm stammt auch die folgende Begründung, weshalb es nicht mehr als die bekannten fünf platonischen Körper geben kann:
\begin{proof}
a) Zunächst wollen wir überlegen, warum reguläre Polyeder nur aus gleichseitigen Dreiecken, Quadraten oder regelmäßigen Pentagonen bestehen kann. Um überhaupt eine Polyederecke herausbilden zu können, werden mindestens drei Polygone benötigt. \\
Weiter können wir festhalten, dass die Winkelsumme der Kanten, die in der Polyederecke aufeinander treffen, kleiner als der Vollwinkel sein muss, um eine konvexe Ecke zu formen. Anderenfalls, würde die Winkelsumme genau $360^\circ$ betragen, würden die Flächen eine Parkettierung der Ebene bilden. Auch bei mehr als $360^\circ$ wäre hingegen keine Ecke möglich.\\
Die Winkelsumme in regulären n-Ecken kann durch die Formel $(n-2)\cdot 180^\circ$ beschrieben werden (Begründung: Durch das Einzeichnen eines Punktes innerhalb des Polygons kann dieses in Dreiecke unterteilt werden. Die Hinzunahme einer weiteren Ecke erlaubt es uns, genau ein weiteres Dreieck einzuzeichnen. So kann die Zunahme der Winkelsumme um $180^\circ$ mit jeder zusätzlichen Ecke erklärt werden.). Wir erhalten für Dreiecke, Quadrate und regelmäßige Pentagone eine Innenwinkelsumme von $180^\circ, 360^\circ$ bzw. $540^\circ$. Die Winkel in den Ecken sind dementsprechend beim Dreieck $60^\circ$, beim Quadrat $90^\circ$ und im regelmäßigen Fünfeck $120^\circ$ groß. \\
Das regelmäßige Hexagon hat eine Innenwinkelsumme von $720^\circ$, die Winkel in den Ecken messen dementsprechend je $120^\circ$. Drei Hexagone bilden also wegen $3\cdot 120^\circ=360^\circ$ keine konvexe Ecke mehr, natürlich bilden auch vier, fünf, etc. Hexagone keine konvexe Ecke. Auf gleiche Weise sehen wir, dass auch das Aneinanderlegen von drei oder mehr Polygonen mit mehr als sechs Ecken keine konvexe Ecke ergibt.\\
Wir halten fest, dass die Flächen der platonischen Körper nur regelmäßige Polygone mit drei, vier oder fünf Ecken sein können.\\

b) Jetzt wollen wir überlegen wie viele Dreiecke, Quadrate oder Pentagone an jeweils einer Ecke des Polyeders zusammentreffen können. Wir orientieren uns dabei an obiger Argumentation.\\
Gleichseitige Dreieck besitzen Innenwinkel von $60^\circ$. Wir können bis zu fünf dieser Dreiecke an ihren Ecke zu einer konvexen Polyederecke zusammenfügen. Die auf diese Weise entstehenden Körperecken die durch das Aufeinandertreffen von drei, vier oder fünf Dreiecken entstehen sind  $180^\circ, 240^\circ$ bzw. $300^\circ$ groß.\\
Quadrate besitzen Innenwinkel von $90^\circ$. Um den Vollwinkel nicht zu überschreiten, können wir nur die Mindestanzahl von drei Flächen aneinanderlegen $(3\cdot)90^\circ=270^\circ)$.\\
Gleiches gilt, wenn wir regelmäßige Pentagone als Seitenflächen wählen: Drei Fünfecke bilden eine konvexe Ecke, die einen Winkel von $324^\circ$ einschließt.\\
Insgesamt erhalten wir drei Polyeder mit dreieckigen Seitenflächen und jeweils einen mit vier- bzw. fünfeckigen Seitenflächen. Genau genommen wissen wir nun erst, dass er nicht mehr als fünf platonische Körper geben kann. Die Existenz haben wir nicht explizit nachgewiesen, jedoch kennen wir die platonischen Körper Tetraeder, Oktaeder und  Ikosaeder (dreieckige Flächen), Hexaeder (viereckige Flächen) und Dodekaeder (fünfeckige Flächen) und damit ist ihre Existenz schließlich auch gesichert. 
\end{proof}
Einige Jahrhunderte später fanden die platonische Körper auch in anderen Disziplinen einen Platz: 
Der Astronom Johannes Kepler konstruierte Ende des 16. Jahrhunderts (1596 in Mysterium Cosmographicum) ineinander verschachtelte Modelle der platonische Körper, um die Umlaufbahnen der bis dato sechs bekannten Planeten (noch unbekannt: Uranus und Neptun) um die Sonne zu beschreiben. Kepler nutzte die Eigenschaft, dass jeder reguläre Polyeder eine Inkugel und eine Umkugel hat. Auf der Inkugel liegen die Mittelpunkte der Seitenflächen des Polyeders, während alle konvexen Ecken auf der Umkugel liegen. Auf diesen Kugelschalen beschreiben die Planeten nach Kepler Kreisbahnen. Die platonischen Körper passte der Astronom so von innen nach außen zwischen den Kugelhüllen ein, dass eine Kugel die Inkugel des Polyeders ist, während die nächste seine Umkugel ist. Dadurch ergab sich folgende Anordnung: Das Oktaeder lag zwischen Merkur und Venus, das Ikosaeder zeischen Venus und Erde, das Dodekaeder zwischen Erde und Mars, das Tetraeder zwischen Mars und Jupiter und das Hexaeder zwischen Jupiter und Saturn. \footnote{http://www.mathe.tu-freiberg.de/~hebisch/cafe/platonische.html}  Wenngleich Keplers Theorie später widerlegt wurde, war er einer der ersten Naturwissenschaftler, der auf geometrische Sachverhalte zur Erklärung astronomischer Phänomene zurückgriff.\\
Leonhard Euler formulierte im 18. Jahrhundert den Eulerschen Polyedersatz, der eine Aussage über das Verhältnis der Anzahlen an Seitenflächen, Kanten und Ecken macht. Wirklich neu ist das Resultat nicht: Vermutlich wussten auch Descartes im 17. Jahrhundert und der Grieche Archimedes bereits von dem Zusammenhang. Im letzten Kapitel dieser Arbeit sind Möglichkeiten aufgezeigt wie der Satz im schulischen Geometrieunterricht eingebunden (und hergeleitet) werden kann. Insbesondere brauchen wir den \textit{Eulerschen Polyedersatz}, um den angekündigten zweiten Beweis zur Existenz der fünf platonischen Körper führen zu können. 
\begin{theorem}
Eulerscher Poleyedersatz. 

\end{theorem}
\begin{proof}
Eulerscher Polyedersatz
\end{proof}



