\subsection{Platonische Körper}
In den vorherigen Kapiteln haben wir uns mit der Klassifizierung der endlichen Untergruppen der orthogonalen Gruppen in zwei Dimensionen beschäftigt. Dabei haben wir feststellen können, dass es für die Klassifizierung ausreicht nur Gruppen zu betrachten, die zu reguläre Polygonen gehören. Wie zum Beispiel die Gruppe $\mathcal{C}_4$ die zu einem Quadrat gehört. Da wir uns nun damit beschäftigen wollen wie wir die endlichen Untergruppen der orthogonalen Gruppe in drei Dimensionen klassifizieren können, liegt es nahe sich mit reguläre Polyedern zu befassen. Im Verlauf dieses Kapitels werden wir sehen, dass es nur fünf reguläre Polyeder geben kann, die fünf platonischen Körper. Bevor wir jedoch den Beweis diskutieren wollen, werfen wir zunächst einen Blick zurück in die Geschichte und befassen uns mit der Entdeckung der platonischen Körper und ihrer Bedeutung im Laufe der Jahrhunderte.\\ \\
Warum beschäftigen wir uns mit den platonischen Körpern?
Motivation: 2dimensional: regelmäßige Polygone
alle Gruppen im 2dimensionalen wirken auf reguläre Polygonen
Wie ist es im 3dimensionalen?
Wir betrachten regelmäßige Polyeder: wir werden sehen, es nur 5 regelmäßige Polyeder gibt. Das sind gerade die platonischen Körper
\\ \\
Bevor wir uns mit dem Beweis des eulerschen Polyederformel beschäftigen müssen wir zunächst klären was einfache, konvexe und reguläre Polyeder sind.
\begin{defi}[Einfaches Polyeder]
	Ein Polyeder heißt einfach, wenn seine Oberfläche sich stetig in eine Kugelfläche überführen lässt, das heißt einfach Polyeder haben keine \enquote{Löcher} \citep[211]{Mainzer1988}
\end{defi}
\begin{defi}[Konvexes Polyeder]
	Ein Polyeder heißt konvex, wenn zu je zwei Punkten aus dem Inneren des Polyeders auch deren Verbindungsstrecke ganz im Polyeder liegt. \citep[51]{Mueller2012}
\end{defi}
\begin{defi}[Reguläres Polyeder]
	Ein konvexes Polyeder heißt regulär, wenn alle Flächen zueinander kongruente regelmäßige Vielecke sind und in jeder Ecke gleich viele Vielecke zusammenstoßen. \citep[51]{Mueller2012}
\end{defi}