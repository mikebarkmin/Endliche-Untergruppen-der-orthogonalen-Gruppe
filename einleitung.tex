\section{Einleitung}
%Eine Einleitung, in der Sie das Thema der Arbeit, die Einordnung in den wissenschaftlichen Kontext, die Geschichte der Fragestellung und die genaue Art Ihrer Bearbeitung (etwa: Ausarbeitung mit eigenem Beispiel, Computersimulation zu Beispiel xy, Ausarbeitung und Beweis des Satzes yz etc) besprechen. Diese Einleitung richtet sich an einen mathematisch vorgebildeten, aber mit der genauen Thematik nicht notwendigerweise vertrauten Leser. Sie geben auch in der Einleitung noch einmal die Textvorlagen an, die Sie verwendet haben.
Die Betrachtung der orthogonalen Gruppe stellt aus didaktischer Sicht eine Brücke zwischen der Linearen Algebra und der Algebra dar: Sie ermöglicht die Anwendung und Interpretation von Begriffen aus der Linearen Algebra und fördert das Verständnis vom Gruppenbegriff durch den Umgang mit nicht-trivialen Gruppen und leistet damit Vorarbeit auf dem Gebiet der Algebra. Insbesondere bieten die orthogonalen Gruppen und ihre Aktionen auf geometrischen Objekten einen mathematischen Zugang zur Symmetrie, was sich wiederum positiv auf die Anschaulichkeit auswirkt.\\
Der Betrachtung der orthogonalen Gruppen im zwei- bzw. dreidimensionalen euklidischen Vektorraum wohnt damit sogar Potential für die Thematisierung im Fachunterricht Mathematik in der Sek. II inne.
%Die vorliegende Arbeit könnte in diesem Kontext beispielsweise als Basisliteratur für die Lehrkraft dienen. Einerseits sind die fachlichen Grundlagen mit Sätzen und Beweisen abgesichert, andererseits wurde Wert auf die Anschaulichkeit der Darstellung gelegt.\\
Zunächst werden wir die orthogonalen Abbildungen im zweidimensionalen Raum betrachten. Wir werden erstens sehen, dass dort jede orthogonale Abbildung eine Drehung oder eine Spiegelung ist. Zweitens können wir eine Klassifikation der endlichen Untergruppen der orthogonalen Gruppen vornehmen. Insbesondere stellen wir fest, dass wir Gruppen mit regulären Polygonen assoziieren können.\\
Darauf aufbauend betrachten wir im nächsten Kapitel die orthogonalen Abbildungen im dreidimensionalen Raum, wo wir neben Spiegelungen und Drehungen zusätzlich Drehspiegelungen unterscheiden können. An die Stelle der Polygone treten nun die regulären Polyeder im Abschnitt zu den Platonischen Körpern. Nach einem kurzen Exkurs zur Rezeption der Platonischen Körper in den Naturwissenschaften seit der Antike werden wir uns auf zwei Arten davon überzeugen, dass es nur fünf regelmäßige Polyeder geben kann. Wir bedienen uns dabei einmal elementargeometrischer Überlegungen, während wir beim zweiten Beweis auf Begriffe der Graphentheorie und den Eulerschen Polyedersatz zurückgreifen.\\
Wie auch im zweidimensionalen Raum nehmen wir eine Klassifikation der endlichen Untergruppen vor: Wir beginnen mit der Klassifikation der endlichen Drehgruppen und beschäftigen uns mit den Drehgruppen der platonischen Körpern. Eine selbstgeschriebene Web-Applikation visualisiert alle Drehungen der platonischen Körper. Schließlich erhalten wir die vollständige Liste der fünf endlichen Untergruppen von Drehungen.\\
Verallgemeinert wird das Resultat durch die Ausweitung der Klassifikation auf die endlichen orthogonalen Gruppen im letzten Unterkapitel. Im letzten Satz werden wir feststellen, dass wir zu jeder Untergruppe der orthogonalen Gruppe im dreidimensionalen Raum eine Klasse angeben können, zu der sie isomorph ist und diese Liste umfassend ist.
Als Literatur für diese Arbeit dient das Buch \enquote{Finite Reflection Groups} von L.C. Grove und C.T Benson.
