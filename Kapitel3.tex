\section{Orthogonale Transformationen im dreidimensionalen Raum}
\begin{theorem}
 Angenommen $T$ ist eine Drehung in $\OR{3}$, dann ist $T$ eine Drehung um eine fixierte Achse, sodass $T$ einen Eigenvektor $x$ zum Eigenwert $1$ besitzt und die Einschränkung von $T$ auf die Ebene $\mathcal{P}=x^{\perp}$ eine Drehung im zweidimensionalen Raum ist.
\end{theorem}
\begin{proof}
 Angenommen $T$ ist eine Drehung in $\OR(3)$ und $\lambda_1,\lambda_2,\lambda_3$ sind Eigenwerte von $T$. Sei $\lambda_1 \in \mathbb{R}$, dann können $\lambda_2$ und $\lambda_3$ nur folgende Werte annehmen, da $\det(T)=\lambda_1\lambda_2\lambda_3=1$.
 \begin{align}
  \lambda_1=1, \lambda_2=\lambda_3=\pm1 \\
  \lambda_1=1, \lambda_2=\overline{\lambda_3}\notin \mathbb{R}
 \end{align}
Für uns ist aber nur der Eigenwert $1$ interessant, dieser ist aber in beiden Fällen vorhanden.

Wir wählen einen Eigenvektor $x$ zum Eigenwert $1$ und bemerken, dass $x=T^{-1}Tx=T^{-1}x$ gilt. Deshalb gilt auch $0=(x^{\perp},x)=(x^{\perp},T^{-1}x)=(Tx^{\perp},x)$ und $\mathcal{P}=x^{\perp}$ ist invariant unter $T$. Betrachten wir nun die Determinate der Einschränkung $T|\mathcal{P}$, dann gilt $\det(T|\mathcal{P})=1$. Demnach ist $T|\mathcal{P}$ eine Drehung der Ebene $\mathcal{P}$
\end{proof}
\begin{bem}
 Wenn $S$ eine Spiegelung an der Ebene $\mathcal{P}$ in $\OR(3)$ ist, dann gilt $Sx=x \ \forall x \in \mathcal{P}$ und $Sy=-y \ \forall y \in \mathcal{P}^{\perp}$. Wir können nun einen Einheitsvektor $r \in \mathcal{P}^{\perp}$ so wählen, dass $Sx=x-2(x,r)r \ \forall x\in \mathbb{R}^3$ gilt. Wenn wir nun eine Basis $\{r,x_2,x_3\}$ von $\mathcal{P}$ wählen, sodass die Transformationsmatirx unter dieser Basis von $S$ durch 
 \begin{center}
  $A= \begin{pmatrix}
        -1 && 0 && 0 \\
        0 && 1 && 0 \\
        0 && 0 && 1 
       \end{pmatrix}$
 \end{center}
repräsentiert wird. Es gilt außerdem $S^2 = E_3$.
\end{bem}
\begin{theorem}
 Sei $T\in\OR{3}$ mit $det T = -1$. Dann ist $T$ eine Drehspiegelung mit Spiegelebene $\mathcal{P}$ und Drehachse $\mathcal{P}^{\perp}$.
\end{theorem}
\begin{proof}
 Angenommen $T$ ist eine Spiegelung in $\OR(3)$ und $\lambda_1,\lambda_2,\lambda_3$ sind Eigenwerte von $T$. Sei $\lambda_1 \in \mathbb{R}$, dann können $\lambda_2$ und $\lambda_3$ nur folgende Werte annehmen, da $\det(T)=\lambda_1\lambda_2\lambda_3=-1$.
 \setcounter{equation}{0}
 \begin{align}
  \lambda_1=-1, \lambda_2=\lambda_3=\pm1 \\
  \lambda_1=-1, \lambda_2=\overline{\lambda_3}\notin \mathbb{R}
 \end{align}
 Wir wählen einen Eigenvektor $x$ zum Eigenwert $-1$ und wählen $\mathcal{P}=x^{\perp}$. Betrachten wir nur die Determinate der Einschränkung $T|\mathcal{P}$, also $det(T|\mathcal{P})=\lambda_2\lambda_3=1$, dann lässt sich erkennen, dass $T|\mathcal{P}$ eine Drehung der Ebene $\mathcal{P}$ sein muss. Wir wählen nun eine Basis $\{x_1,x_2\}$ von $\mathcal{P}$, sodass die Transformationsmatirx von $T$ unter der Basis $\{x_1,x_2,x_3\}$ durch \begin{align*}
  A&=
\begin{pmatrix}
        -1 && 0 && 0 \\
        0 && \cos(\theta) && -\sin(\theta) \\
        0 && \sin(\theta) && \cos(\theta)
       \end{pmatrix} \\ &=
       \begin{pmatrix}
        1 && 0 && 0 \\
        0 && \cos(\theta) && -\sin(\theta) \\
        0 && \sin(\theta) && \cos(\theta)
       \end{pmatrix} 
       \begin{pmatrix}
        -1 && 0 && 0 \\
        0 && 1 && 0 \\
        0 && 0 && 1
       \end{pmatrix}
 \end{align*}
repräsentiert wird.
\end{proof}



