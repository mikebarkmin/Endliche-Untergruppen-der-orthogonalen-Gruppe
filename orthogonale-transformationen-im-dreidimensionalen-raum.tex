\section{Orthogonale Abbildungen im dreidimensionalen Raum}
Wir betrachten nacheinander Drehungen, Spiegelungen und Drehspiegelungen und charakterisieren diese, indem wir ihre Eigenschaften nennen und durch Beweis begründen.
\begin{theorem}
Sei $T$ eine Drehung aus der orthogonalen Gruppe $\OR{3}$, dann hat $T$ eine fixierte Drehachse, die durch den Eigenvektor $x$ von $T$ zum Eigenwert 1 eindeutig festgelegt ist.
Wenn wir $T$ auf die durch $x^{\perp}$ eindeutig festgelegte Ebene $\mathcal{P}$ einschränken, ist $T$ eine Drehung im zweidimensionalen Raum $\mathbb{R}^2$.
\end{theorem}
\begin{proof}
Zunächst wollen wir klären, wie der Eigenwert 1 zustande kommt:
 Angenommen $T$ ist eine Drehung in $\OR{3}$ und $\lambda_1,\lambda_2,\lambda_3$ sind Eigenwerte von $T$, dann muss mindestens einer der Koeffizienten aus $\mathbb{R}$ sein, da nach dem Zwischenwertsatz aus der Analysis ein reelles Polynom dritten Grades mindestens eine reelle Nullstelle besitzt. Sei $\lambda_1$ (ggf. nach Umbenennung) aus $\mathbb{R}$, dann können $\lambda_2$ und $\lambda_3$ nur folgende Werte annehmen, da $\det(T)=\lambda_1\lambda_2\lambda_3=1$.
 \begin{align}
  \lambda_1=1, \lambda_2=\lambda_3=\pm1 \\
  \lambda_1=1, \lambda_2=\overline{\lambda_3}\notin \mathbb{R}
 \end{align}
Für uns ist nur der Eigenwert $1$ interessant, dieser ist aber in beiden Fällen vorhanden.\\
Es bleibt nachzuweisen, weshalb $\mathcal{T}|_\mathcal{P}$ eine Drehung in der Ebene $\mathcal{P}$ ist.
Sei $x$ ein Eigenvektor zum Eigenwert $1$, dann gilt nach der Definition eines Eigenvektors: $x=T^{-1}Tx=T^{-1}x$. Daraus können wir folgende Gleichungskette ableiten: $0=(x^{\perp},x)=(x^{\perp},T^{-1}x)=(Tx^{\perp},x)$ und direkt ablesen, dass $\mathcal{P}=x^{\perp}$ invariant ist unter $T$. Betrachten wir nun die Determinate der Einschränkung $T|_\mathcal{P}$, dann gilt $\det(T|_\mathcal{P})=1$, wegen $\det(T|_\mathcal{P}) = \lambda_2\lambda_3$ und $\lambda_2,\lambda_3$ wie oben.
\end{proof}
Die Eigenschaften der Spiegelungen in $\OR{3}$ fassen wir in einer Bemerkung zusammen:
\begin{bem}
    Eine Spiegelung $S$ an der Ebene $\mathcal{P}$ ist eine Abbildung, für die gilt, dass $Sx=x$ für alle $x \in \mathcal{P}$. Die $y$ aus dem orthogonalen Komplement $\mathcal{P}^{\perp}$ der Ebene $\mathcal{P}$ werden auf ihr Negatives abgebildet, kurz $Sy=-y$ für alle $y\in \mathcal{P}$.\\
    Für alle Vektoren $z \in \mathbb{R}^3$ gibt es einen Einheitsvektor $r$ aus $\mathcal{P}^{\perp}$, sodass die Abbildungsvorschrift $Sz = z -2(z,r)r$ gilt.\\
Nach Wahl einer geeigneten Basis $\{r, x_2, x_3\}$, wobei $\{x_2, x_3\}$ Basis der Spiegelebene $\mathcal{P}$ ist, kann die Spiegelung $S$ durch die Matrix $A$ dargestellt werden.
\begin{center}
  $A= \begin{pmatrix}
        -1 && 0 && 0 \\
        0 && 1 && 0 \\
        0 && 0 && 1
       \end{pmatrix}$
 \end{center}
 \end{bem}
 Zweimaliges Spiegeln eines Vektors $z$ aus $\mathbb{R}^3$ überführt den Vektor wieder in sich selbst wegen: $A^2 = S^2 = E_3 = id$.\\
 Zuletzt betrachten wir Drehspiegelungen als Kombination von Spiegelung und Drehung.
 \begin{theorem}
 Sei $T\in\OR{3}$ mit $\det T = -1$. Dann ist $T$ eine Drehspiegelung mit Spiegelebene $\mathcal{P}$ und Drehachse $\mathcal{P}^{\perp}$.
 \end{theorem}
 Der Beweis verläuft ähnlich wie oben.
 \begin{proof}
 Diesmal sei $T$ eine Spiegelung aus $\OR{3}$ und $\lambda_1,\lambda_2, \lambda_3$ die Eigenwerte von $T$. Wie oben gibt es unter der Voraussetzung, dass $\lambda_1 \in \mathbb{R}$  und $\det T = \lambda_1\lambda_2\lambda_3=-1$ gilt, nur zwei mögliche Fälle:
 \begin{align}
   \lambda_1=-1, \lambda_2=\lambda_3=\pm1 \\
   \lambda_1=-1, \lambda_2=\overline{\lambda_3}\notin \mathbb{R}
  \end{align}
 Sei von nun an $x$ ein Eigenvektor zum Eigenwert $\lambda_1=-1$ und $\mathcal{P}$ das orthogonale Komplement $x^{\perp}$. Dass die Einschränkung $T|_\mathcal{P}$ eine Drehung in der Ebene $\mathcal{P}$ ist, wissen wir wegen $\det(T|_\mathcal{P}) = \lambda_2\lambda_3=1.$
 \end{proof}
 Wir wählen nun eine geeignete Basis $\{x_1,x_2\}$ zu $\mathcal{P}$, sodass wir $T$ darstellen können durch die Matrix $A$
 \begin{center}
   $A= \begin{pmatrix}
         -1 && 0 && 0 \\
         0 && \cos(\theta) && -\sin(\theta) \\
         0 && \sin(\theta) && \cos(\theta)
        \end{pmatrix}$.
  \end{center}
 Die Darstellung als Produkt von einer Drehung und einer Spiegelung verdeutlicht den dargelegten Zusammenhang:
 \begin{center}
  $A=\begin{pmatrix}
            1 && 0 && 0 \\
            0 && \cos(\theta) && -\sin(\theta) \\
            0 && \sin(\theta) && \cos(\theta)
           \end{pmatrix}
           \begin{pmatrix}
            -1 && 0 && 0 \\
            0 && 1 && 0 \\
            0 && 0 && 1
           \end{pmatrix}$.
 \end{center}



